

\pdfminorversion 7
\pdfobjcompresslevel 3

\documentclass[a4paper]{article}
\special{papersize=210mm,297mm}
\usepackage[utf8]{inputenc}
\usepackage[T1]{fontenc}
\usepackage{cite}
\usepackage[francais]{babel}
\usepackage[bookmarks=false,colorlinks,linkcolor=blue]{hyperref}
\usepackage[top=3cm,bottom=2cm,left=3cm,right=2cm]{geometry}
\usepackage{graphicx}
\usepackage{subfig}
\usepackage{eso-pic}
\usepackage{array}
\usepackage{color}
\usepackage{url}
\usepackage{listings}
\usepackage{eurosym}
\usepackage{url}
\usepackage{textcomp}
\usepackage{fancyhdr} 
\usepackage{tikz}
\usetikzlibrary{automata,positioning}

\definecolor{lightgray}{gray}{0.9}

\title{Projet de SACC}
\author{Rémy \textsc{El-Sibaie Besognet} -- Roven \textsc{Gabriel}}

\newcommand{\HRule}{\rule{\linewidth}{0.5mm}}


\begin{document}

\maketitle

\section{Introduction}

Le logiciel Pave est un interprète pour un langage de modélisation de
système concurrent : CCS\footnote{Calculus of communicating
systems}. Il a été développé par les différentes générations
d'étudiants de l'UE SACC. La dernière pierre ajoutée à cet édifice est
la prise en main du $\mu$-Calcul dans l'interprète. Après l'ajout de
sucre syntaxique, l'objectif est d'implémenter un algorithme de
vérification d'un processus décrit en CCS à partir d'une formule
$\mu$-Calcul.

On appelle cette vérification du \emph{Model Checking}. On vérifie
que la description d'un programme est fidèle à sa spécification de
façon automatique. On présentera ici deux algorithme pour le
Model Checking
à hauteur de notre compréhension : local et global. Seul
l'algorithme local a été écrit dans la version proposée par notre binôme.

%% \section{$\mu$-Calcul par valeur}

\section{Modifications préliminaires}

Avant de pouvoir implémenter la vérification de formules, nous avons effectués
quelques modifications dans l'optique d'améliorer l'utilisation de \texttt{pave}.
Nous avons amélioré l'affichage d'erreur qui non seulement n'arrête pas le 
programme mais aussi est plus clair. 

\begin{verbatim}
> prop A = <a!>;
               ^
Parser error at line 1 char 14: ~;~
>
\end{verbatim}

Nous avons modifié le parseur afin de pouvoir définir et vérifier des formules du
$\mu$-calcul et défini l'AST représentant  les formules. Nous avons deux
particularités notables : les modalités sont représentées sous forme de triplet
\emph{(force, type, restriction)} et les opérations \texttt{Mu} et \texttt{Nu}
embarquent directement l'environement qui sera utilisé dans le \texttt{check local}.

\section{Model checking local}

L'algorithme de model checking local a été inspiré par Glynn Winskel
dans \emph{Topics in concurrency}. Le principe est en fait assez
simple. On parcourt récursivement le processus et la formule en
vérifiant que chaque état du processus vérifie la formule. On traite
les cas triviaux en traduisant simplement l'opérateur $\mu$-Calcul en
opérateur booléen inductivement.

\begin{lstlisting}[language=caml]
let rec check def_map prop_map formula nproc =
  let rec check_internal = function
    | FTrue -> true
    | FFalse -> false
    | FNot formula -> not @@ check_internal formula
    | FAnd (f1, f2) -> check_internal f1 && check_internal f2
    | FOr (f1, f2) -> check_internal f1 || check_internal f2
    | FImplies (f1, f2) -> check_internal f1 |> not || check_internal f2
    | ...
\end{lstlisting}

On considère un processus $p$. Dans le cas des modalités, on récupère
toutes les dérivations possibles de $p$ dont la transition est
étiquetée par la bonne action (sans oublier \texttt{<>}
et \texttt{[]}) et on construit un ensemble de processus qui
correspond aux suivants de $p$. On appelle récursivement notre
fonction \texttt{check} sur les éléments de cet ensemble avec la
formule sans la modalité.

La partie concernant les points fixes n'était pas triviale. L'idée est
de définir une fonction de \emph{subtitution} et de remplacer
récursivement le résultat courant dans l'appel suivant.

\begin{lstlisting}[language=caml]
let beta_reduce in_formula expected_var replacement =
  let rec beta_reduce in_formula =
  match in_formula with
  | FTrue | FFalse -> in_formula
  | FAnd (f1, f2) -> FAnd(beta_reduce f1, beta_reduce f2)
  | FOr (f1, f2) -> FOr(beta_reduce f1, beta_reduce f2)
  | FImplies (f1, f2) -> FImplies(beta_reduce f1, beta_reduce f2)
  | FModal (modality, formula) -> FModal(modality, beta_reduce formula)
  | FInvModal (modality, formula) -> FInvModal(modality, beta_reduce formula)
  | FProp _ -> in_formula
  | FVar var when var = expected_var -> replacement
  | FVar _ -> in_formula
  | FMu (x, env, formula) -> FMu(x, env, beta_reduce formula)
  | FNu (x, env, formula) -> FNu(x, env, beta_reduce formula)
  | FNot formula -> FNot (beta_reduce formula)
  in
  beta_reduce in_formula
\end{lstlisting}

\subsection{Trace}

Afin de mieux comprendre les raison d'une propriété fausse, nous avons
ajouté la fonctionnalitée de trace. Ainsi, à chaque étape dans
l'algorithme, la formule courante et processus courant sont stockés et
restitués à la fin de l'opération. Ainsi nous obtenons le chemin qui
arrive à la propriété fausse.

\begin{verbatim}
checklocal ~<a?>(<a!>true) |- a!, a?;
Trace : 
	a!,a?,0 -| ~<a?><a!>True
	a?,0 -| <a!>True
FALSE PROPERTY
\end{verbatim}

Dans cet exemple nous pouvons voir que c'est à
l'étape \verb|a?,0| associé à la
formule \verb|<a!>True| que la propriété devient fausse.

\begin{verbatim}
def P = a!,P + b!,P;
def D = P || c!;

checklocal <a!><b!><a!><a!><c!><b!><b!><c!><c!><c!>true |- D;
Trace : 
	D() -| <a!><b!><a!><a!><c!><b!><b!><c!><c!><c!>True
	(c!,0||P()) -| <b!><a!><a!><c!><b!><b!><c!><c!><c!>True
	(c!,0||P()) -| <a!><a!><c!><b!><b!><c!><c!><c!>True
	(c!,0||P()) -| <a!><c!><b!><b!><c!><c!><c!>True
	(c!,0||P()) -| <c!><b!><b!><c!><c!><c!>True
	P() -| <b!><b!><c!><c!><c!>True
	P() -| <b!><c!><c!><c!>True
	P() -| <c!><c!><c!>True
FALSE PROPERTY
\end{verbatim}

Un autre exemple de trace qui nous permet de voir que \verb|c!| ne
peut apparaitre qu'une seule fois, la consomation de cette action est
clairement visible sur les processus utilisés à gauche.

\section{Model Checking global}

L'algorithme global pour le Model Checking a, lui, été inspiré par
Sergey Berezin dans \emph{Model Checking algorithms for
$\mu$-Calculus} L'implémentation de cet algorithme dans le projet a
été arrêtée par manque de compréhension de l'article présenté et de
l'idée même de la méthode dite globale. Nous n'avons pas compris les
paramètres et ce qu'était sensé rendre l'algorithme.

Notre première démarche a été de vouloir le résoudre avec la méthode
proposée des BDD, d'où la présence du module correspondant. La
traduction d'une formule vers un BDD demandait préalablement d'avoir
compris l'algorithme simple. Nous avons donc tenté cette approche sans
plus de succès.


\section{Exemples}

\subsection{Définition de propriétés usuelles}
\begin{verbatim}
prop Exercice5 = Mu(X).<a!>true or (<.>true and [.]X):
prop Possibly(A) = Mu(X).A or <.>X;
prop Deadlock = [.]false;
prop Always(A) = Nu(X).A and [.]X;
prop Continue = <.>true;
prop Eventualy(A) = Mu(X).A or ([.]X and <.> true);
\end{verbatim}

\subsection{Vérifications}
\begin{verbatim}
def D2 = a!,(b? + D2);
\end{verbatim}

\begin{verbatim}
> checklocal Always(Continue) |- D2;
Trace : 
	D2() -| Always(Continue)
	D2() -| Nu(X){()}.(Continue and [.]X)
	D2() -| (Continue and [.]Nu(X){(D2())}.(Continue and [.]X))
	D2() -| Continue
	D2() -| <.>True
	(b?,0+D2()) -| True
	D2() -| [.]Nu(X){(D2())}.(Continue and [.]X)
	(b?,0+D2()) -| Nu(X){(D2())}.(Continue and [.]X)
	(b?,0+D2()) -| (Continue and [.]Nu(X){((b?,0+D2()),D2())}.(Continue and [.]X))
	(b?,0+D2()) -| Continue
	(b?,0+D2()) -| <.>True
	0 -| True
	(b?,0+D2()) -| True
	(b?,0+D2()) -| [.]Nu(X){((b?,0+D2()),D2())}.(Continue and [.]X)
	0 -| Nu(X){((b?,0+D2()),D2())}.(Continue and [.]X)
	0 -| (Continue and [.]Nu(X){(0,(b?,0+D2()),D2())}.(Continue and [.]X))
	0 -| Continue
	0 -| <.>True
	(b?,0+D2()) -| Nu(X){((b?,0+D2()),D2())}.(Continue and [.]X)
FALSE PROPERTY
\end{verbatim}

Le processus ne s'execute pas à l'infini tout le temps.

\begin{verbatim}
> checklocal Possibly(Continue) |- D2;
TRUE PROPERTY

> checklocal Possibly(Deadlock) |- D2;
TRUE PROPERTY
\end{verbatim}

Il est cependant possible qu'il s'execute à l'infini ou qu'il s'arrête.

\begin{verbatim}
> checklocal Always(Deadlock) |- D2;
Trace : 
	D2() -| Always(Deadlock)
	D2() -| Nu(X){()}.(Deadlock and [.]X)
	D2() -| (Deadlock and [.]Nu(X){(D2())}.(Deadlock and [.]X))
	D2() -| Deadlock
	D2() -| [.]False
	(b?,0+D2()) -| False
FALSE PROPERTY

\end{verbatim}

Il ne s'arrête donc pas tout le temps.

\begin{verbatim}
def E = d!,c!,e!;
def F = (c!,b!,a!,b! || c!,a!,b!);
def G = (c!,b!,a!,b! || c!,a!,b! || E);

prop SuiteAAC = Possibly(<a!><a!><c!>true);

> checklocal SuiteAAC |- F;
Trace : 
	F() -| SuiteAAC
	F() -| Possibly(<a!><a!><c!>True)
	F() -| Mu(X){()}.(<a!><a!><c!>True or <.>X)
	F() -| not Nu(X){()}.not (<a!><a!><c!>True or <.>not X)

[...]
	0 -| <.>not Nu(X){(0,b!,0,(b!,0||b!,0),(a!,b!,0||b!,0),
            (b!,0||c!,a!,b!,0),(a!,b!,0||c!,a!,b!,0),(b!,a!,b!,0||c!,a!,b!,0),F()a}.
            not (<a!><a!><c!>True or <.>not X)
FALSE PROPERTY

> checklocal SuiteAAC |- G;
TRUE PROPERTY
\end{verbatim}

Dans cet exemple nous pouvons observer que la
séquence \texttt{a!,a!,c!} n'est possible dans le processus F que si
un troisième processus parallel E contenant
\texttt{c!} est défini.

\end{document}

# Local Variables:
# compile-command: "rubber -d rapport.tex"
# End:

%%  LocalWords:  Glynn Winskel Topics concurrency
